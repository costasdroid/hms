\documentclass[greek]{beamer}
%\usepackage{fontspec}
%\newtheorem{definition}{Ορισμός}
\setbeamercovered{transparent}
\usepackage{amsmath,amsthm}
\usepackage{unicode-math}
\usepackage{xltxtra}
\usepackage{graphicx}
\usetheme{Warsaw}
\usecolortheme{seahorse}
\usepackage{hyperref}
\usepackage{ulem}
\usepackage{xgreek}
\usepackage{pgfpages}
%\setbeameroption{show notes on second screen}
%\setbeameroption{show only notes}

\setsansfont{Times New Roman}
\title{Γράφω Μαθηματικά στον Υπολογιστή}
\author[Λόλας, Πετρίδης]{Κ. Λόλας\inst{1} \and Π. Πετρίδης\inst{2}}
\institute[]
{
  \inst{1}%
  10ο ΓΕΛ ΘΕΣ/ΝΙΚΗΣ (ΠΕ03)
  \and
  \inst{2}%
  ΓΕΛ ΧΑΛΑΣΤΡΑΣ (ΠΕ04.01)
}
\date{Αθήνα, Δεκέμβριος 2018}

\begin{document}
\begin{frame}
  \titlepage
\end{frame}

\section{Εισαγωγή}
\begin{frame}{Γνωριμία}
  \begin{itemize}
    \item Word - MathType
      \begin{itemize}
        \item Βιβλίο Άλγεβρας ΑΠΘ ~ 2001
        \item Εργασίες για φοιτητές 2000 - σήμερα
        \item Δημοσιεύσεις στην Μαθηματική Εταιρεία
      \end{itemize}
    \item Scientific Workplace
      \begin{itemize}
        \item Τράπεζα Ασκήσεων ~ 2001
      \end{itemize}
    \item MathJax
      \begin{itemize}
        \item Blogs, CMSs 2015 - σήμερα
      \end{itemize}
    \item \LaTeX
      \begin{itemize}
        \item Όλα τα αρχεία 2010 - σήμερα
      \end{itemize}
  \end{itemize}
\end{frame}

\subsection{Περί}
\begin{frame}{Καλύτερο?}
  \begin{center}
    \begin{block}{}
      \begin{center}Δεν υπάρχει ΤΟ καλύτερο!\end{center}
    \end{block}
  \end{center}
  \note{Στο βιβλίο της Β Γυμνασίου υπάρχει ο ορισμός... \\ Δεν αφήνει περιθώρια για ερμηνεία...}
\end{frame}

\subsection{Κατηγορίες}
\begin{frame}{Κατηγορίες}
  % \begin{figure}
  %   \includegraphics[scale=0.2]{}
  % \end{figure}
  \begin{itemize}
    \item<1> Τύπος Λογισμικού
    \only<1>{
      \begin{itemize}
        \item Ανοιχτό
        \item Κλειστό
      \end{itemize}
      }
    \item<2> Ελεύθερο
    \item<3> Τύπος Αρχείου
      \only<3>{
        \begin{itemize}
          \item Κωδικοποιημένο
          \item Επεξεργάσιμο
        \end{itemize}
        }
    \item<4> Περιβάλλον
      \only<4>{
        \begin{itemize}
          \item Κειμενογράφος
          \item WYSIWYG
        \end{itemize}
        }
    \item<5> Συνεργατικό
      \only<5>{
        \begin{itemize}
          \item Στατικό
          \item Επεξεργάσιμο
        \end{itemize}
        }
    \item<6> Τύπος Τελικής Μορφής
      \only<6>{
        \begin{itemize}
          \item pdf, ps ...
          \item ίδιο με το αρχικό
        \end{itemize}
        }
  \end{itemize}
  \note{Διαισθητικά η πίεση είναι μονόμετρο... \\ Ένα βαρόμετρο τοίχου}
\end{frame}

\subsection{Ανάλυση}
\begin{frame}{Word - MathType}
  \begin{itemize}
    \item Κλειστό
    \item Κόστος 299 Word \& 39,95 MathType (4/12/2018) ή όχι
    \item Κωδικοποιημένο αρχείο (μερικώς)
    \item WYSIWYG
    \item Στατικό
    \item Ίδιο με το αρχικό
  \end{itemize}
\end{frame}

\begin{frame}{Word - MathType}
  \begin{itemize}
    \item Υπέρ
      \begin{itemize}
        \item<2-> Σούπερ διαδεδομένο
        \item<3-> WYSIWYG
        \item<4-> Κουμπάκια
      \end{itemize}
    \item Κατά
      \begin{itemize}
        \item<5-> Κόστος
        \item<6-> Συμβατότητα
        \item<7-> Bugs
      \end{itemize}
  \end{itemize}
\end{frame}

\begin{frame}{MathJax}
  \begin{itemize}
    \item Apache License 2.0
    \item δωρεάν
    \item αρχείο html
    \item οποιοσδήποτε κειμενογράφος
    \item συνεργατικό
    \item ίδιο με το αρχικό
  \end{itemize}
\end{frame}

\begin{frame}{MathJax}
  \begin{itemize}
    \item Υπέρ
      \begin{itemize}
        \item<2-> Γρήγορη δημιουργία
        \item<3-> Δεν απαιτείται εγκατάσταση
        \item<4-> Έτοιμη ιστοσελίδα
        \item<5-> Ανοίγει παντού
      \end{itemize}
    \item Κατά
      \begin{itemize}
        \item<6-> MathJax + HTML
        \item<7-> Μία τουλάχιστον φορά σύνδεση internet
      \end{itemize}
  \end{itemize}
\end{frame}

\begin{frame}{LyX}
  \begin{itemize}
    \item GNU General Public License, version 2
    \item δωρεάν
    \item αρχείο κειμένου
    \item προβολή μόνο με Lyx / επεξεργασία οπουδήποτε
    \item τοπικό
    \item ίδιο με το αρχικό
  \end{itemize}
\end{frame}

\begin{frame}{Lyx}
  \begin{itemize}
    \item Υπέρ
      \begin{itemize}
        \item<2-> Πλήρης υποστήριξη \LaTeX
        \item<3-> Πλήρης παλέτα μαθηματικών
        \item<4-> WYSIWYM
      \end{itemize}
    \item Κατά
      \begin{itemize}
        \item<6-> Ελληνικά
        \item<7-> Πολλαπλές οθόνες
      \end{itemize}
  \end{itemize}
\end{frame}

\section{Τα καλύτερα}
\begin{frame}{Θέλω Τώρα!}
  \begin{itemize}
    \item Σε δικό μου PC?
    \item Σε φίλου?
  \end{itemize}
\end{frame}

\begin{frame}{Θέλω Ταυτόχρονα}
  \begin{itemize}
    \item Google Docs
    \item Overleaf
  \end{itemize}
\end{frame}

\begin{frame}{Θέλω να μου προτείνουν αλλαγές}
  \begin{itemize}
    \item github - latex
  \end{itemize}
\end{frame}

\begin{frame}{Θέλω blog}
  \begin{itemize}
    \item jekyll - mathjax - markdown
  \end{itemize}
\end{frame}

\begin{frame}{Θέλω να νιώθω}
  \begin{itemize}
    \item \LaTeX
  \end{itemize}
\end{frame}

\begin{frame}[plain,c]
  \begin{center}
    \Huge Σας Ευχαριστούμε...
  \end{center}
  $$B > \frac{1}{k}\sum_{i=1}^{k}x_k$$
\end{frame}

\end{document}
