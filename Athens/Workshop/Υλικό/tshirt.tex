\documentclass[svgnames]{article}


\usepackage{fontspec,microtype,shapepar,xcolor}
\usepackage{amsmath,amsthm}
\usepackage{unicode-math}
\usepackage{xltxtra}
\usepackage{xgreek}

\setmainfont{Times New Roman}
\begin{document}
\textcolor{Crimson}{\heartpar{Ένας εύκολος τρόπος να λυθεί μία εξίσωση $αx^2+βx+γ=0,α\ne 0$ είναι να υπολογίσουμε την διακρίνουσα $Δ=β^2-4αγ$. Αν
* $Δ>0$ η εξίσωση έχει δύο πραγματικές ρίζες στο $\mathbb{R}$ τις $x_{1,2}=\frac{-β\pm \sqrt{Δ}}{2α}$
* $Δ=0$ η εξίσωση έχει μία διπλή πραγματική ρίζα την $x=-\frac{β}{2α}$
* $Δ< 0$ η εξίσωση έχει δύο μιγαδικές ρίζες τις $z_{1,2}=\frac{-β\pm i\sqrt{-Δ}}{2α}$.
Σε κάθε περίπτωση, αν $x_{1,2}$ είναι οι ρίζες τις εξίσωσης τότε $x_1+x_2=-\frac{β}{α}$ και $x_1\cdot x_2=\frac{γ}{α}$. Οι τύποι αυτοί ονομάστηκαν τύποι του Viete.
}}
\end{document}
